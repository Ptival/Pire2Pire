\chapter{Introduction}
Ce projet consiste en la réalisation d'un logiciel d'échange de fichiers selon
le concept "peer to peer". Nous allons vous présenter dans ce rapport 
l'avancement de notre projet à la sixième semaine.\\
\\
Notre choix d'architecture comportant un daemon et un client, nous allons vous
décrire les fonctionnalités de chaque partie.\\
Tout d'abord le client, qui fait le lien entre l'utilisateur et le daemon. Nous
détaillerons dans cette partie son architecture, les requêtes demandées par le
protocole et enfin l'état actuel de leur implémentation.\\
Ensuite le daemon, qui est appelé ainsi car l'utilisateur n'y a pas accès
directement.
%%ma définition n'est peut être pas la bonne... si vous voulez la changer...
Nous expliciterons son architecture aussi puis nous aborderons des points 
techniques tels que la gestion du multithreading et des fichiers. Nous 
aborderons enfin les protocoles clients-daemon et inter-daemons.
