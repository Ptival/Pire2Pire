\chapter{Introduction}
	Ce projet consiste en la réalisation d'un logiciel d'échange de fichiers 
	selon le concept « peer to peer ». Nous présentons dans ce rapport 
	l'avancement de notre projet à la sixième semaine.\\
	
	Notre choix d'architecture se base sur la possiblité d'agir en multi-utilisateurs 
	sur une machine, à travers une interface commune de communication vers le réseau.
	Notre client peer-to-peer se décompose donc en un serveur local de fichiers 
	(que nous désignerons dans ce rapport par le terme \textbf{« démon »}), et un certain 
	nombre de processus utilisateurs (que nous nommerons des \textbf{« clients »} )
	connectés à cette interface réseau commune.\\
	
	Nous présenterons d'abord le processus client, et détaillerons dans cette partie son 
	architecture, les commandes utilisables par l'utilisateur à travers ce client, et
	l'état actuel de l'implémentation.\\
	
	On détaillera ensuite le rôle du démon que nous avons mis en place.
	Nous expliciterons son architecture puis nous aborderons des points 
	techniques tels que la gestion du multithreading et des fichiers. Nous 
	aborderons enfin les protocoles inter-démons d'échanges de données et 
	de requêtes. 
