%%%%%%%%%%%%%%%%%%%%%%%%%%%%%%%%%%%%%%%%%%%
%                                         %
% This file should only be made of \input %
%                                         %
%%%%%%%%%%%%%%%%%%%%%%%%%%%%%%%%%%%%%%%%%%%

\chapter{Le démon}
    Nous avons choisi de développer le serveur sous forme d'un démon.
S'éxécutant en arrière-plan, il peut à tout moment être contacté par un client
ou par un autre serveur utilisant le même protocole de communication.

    \section{Gestion des clients}
    Dès qu'un client se connecte au démon, un thread est lancé et le gère
jusqu'à ce qu'il se déconnecte (cf. handle\_client () dans
\url{daemon/client_handler.c}). Il est ajouté à une liste doublement chaînée de 
clients connus, afin que l'application ait à tout moment une vision globale des 
clients qui lui sont affectés. 

\begin{lstlisting}
struct client {
    int                     socket;
    sem_t                   socket_lock;
    char                    *addr;
    pthread_t               thread_id;
    struct client_request   *requests;
    sem_t                   req_lock;
    struct client           *next;
    struct client           *prev;
};
\end{lstlisting}

    On remarque ici qu'il est nécessaire d'utiliser des sémaphores afin d'éviter
les "situations de compétition". Un client pourrait en effet avoir besoin de
modifier son champ "requests", et a besoin d'être sûr qu'une seule modification
puisse être effectuée à la fois.

    Dès qu'un client envoie une commande au démon via sa socket, une nouvelle
"struct client\_request" est créée. Elle est ajoutée à la liste doublement
chaînée contenant toutes les requêtes actives pour le client. Un thread est
lancé pour traiter cette requête, et appelle la fonction correspondante dans
\url{daemon/client_requests/}.



\begin{lstlisting}
struct client_request {
    char                    *cmd;
    struct client           *client;
    struct client_request   *prev;
    struct client_request   *next;
    pthread_t               thread_id;
};
\end{lstlisting}


	\section{Architecture du daemon}
	
	\section{Gestion du multithreading}
	
	\section{Gestion des fichiers}
	
    \section{Protocole client/daemon}
    Nous allons maintenant vous présenter les différentes requêtes du client et leur état actuel.

\begin{itemize}
\item{\textbf{set [option=value]}} Si "option=valeur" est spécifié, modifie la valeur
d'une option. Sinon, affiche la liste des options disponibles. Cette commande
est implémentée, mais aucune option n'existe.

\item{\textbf{help}} Affiche une aide indiquant la liste des options
disponibles. Implémentée.

\item{\textbf{list [options]}} Donne la liste des ressources disponibles sur
le réseau sous la forme suivante : \textit{clé nom taille}.

\item{\textbf{get clé}} Récupère la ressource \textit{clé}.

\item{\textbf{info}} affiche des informations sur l'état du programme sous la
forme suivante :

There [is/are] n client(s) connected.     \\
There [is/are] n daemon(s) connected.     \\
You have n request(s) currently running : \\
1 - CMD1                                  \\ 
2 - CMD2                                  \\
...

\item{\textbf{download}} affiche la liste des fichiers en cours de récupération, non encore implémentée.

\item{\textbf{upload}} affiche la liste des fichiers en cours de transfert, non encore implémentée.

\item{\textbf{connect ip:port}} demande au démon de se connecter à un autre
démon. Implémentée.

\item{\textbf{raw ip:port cmd}} envoie la commande de protocole cmd à un autre programme.

\end{itemize} 
	
	\section{Protocoles inter-daemons}
Abordons à présent les requêtes inter-daemons et leur état actuel.

\begin{itemize}
\item{\textbf{list [options]}} Donne la liste des ressources disponibles localement. Implémentée mais sans option pour l'instant.
\item{\textbf{file nom-de-fichier cle taille ip:port}} donne les informations sur le fichier \textit{nom-de-fichier}.
\item{\textbf{get clé début fin}} demande à recevoir les octets compris entre \textit{début} et \textit{fin} du fichier ayant pour md5 \textit{clé}. Non implémentée
\item{\textbf{traffic}} demande la liste des transmissions prévues ou en cours. Non encore implémentée.
\item{\textbf{ready clé delay ip port proto debut fin}} indique une transmission prévue dans \textit{delay} secondes du fragment compris entre \textit{début} et \textit{fin} du fichier identifié par \textit{clé}. Non encore implémentée.
\item{\textbf{checksum clé taille-bloc début fin checksum checksum...}} donne la liste des checksum des blocs de \textit{taille-bloc} octets du fichier de clé \textit{clé} et compris entre \textit{début} et \textit{fin}. Non encore implémentée.
\item{\textbf{neighbourhood}} demande une liste de voisins connus. Implémentée.
\item{\textbf{neighbour ip:port}} Fournit l'adresse et le port d'un membre du réseau.
\item{\textbf{error requête [message]}}Signale un erreur pouvant entrainer une requête \textit{requête} et pouvant être suivie d'un message explicatif \textit{message}
\end{itemize}

	\paragraph*{}conclusion
