\section{Etat du projet}

%%%%%
\subsection{Commandes implémentées}

\subsubsection{Commandes de protocole}
\begin{description}
    \item[list] récupère la liste des fichiers présents sur les machines
connues.
    \item[file fichier hash taille ip:port] uh ?
    \item[get hash debut fin] 
    \item[ready hash délai ip port protocole début fin]
    \item[neighbourhood] demande une liste de voisins connus.
    \item[neighbour ip:port] fournit l'adresse IP et le port TCP d'un des
membres du réseau.
    \item[error] euh, bonne syntaxe ?
\end{description}



\subsubsection{Commandes utilisateur}
\begin{description}
    \item[set] fournit la liste des options disponibles (modifie une option avec
la syntaxe set a=b). Aucune option n'est disponible...
    \item[help] fournit la liste des commandes existantes.
    \item[list] liste les fichiers présents sur les machines auxquelles le
client est connecté. Aucune option n'est implémentée pour cette commande, elle
retournera toujours la totalité des fichiers.
    \item[get hash] télécharge le fichier dont le hash est passé en paramètre. 
    \item[info] donne des informations concernant le programme (nombre de
machines connues, nombre de clients connectés).
    \item[download] donne des informations sur les téléchargements en cours.
    \item[upload] donne des informations sur les envois en cours.
    \item[connect ip:port] se connecte à une autre machine.
    \item[raw ip:port cmd] envoie la commande de protocole "cmd" à la machine
dont l'ip/port sont passés en paramètre.
\end{description}



%%%%%
\subsection{Commandes non implémentées}
\begin{description}
    \item[traffic]
    \item[checksum hash taille-bloc début fin checksum] 
    \item[redirect hash ip:port]

\end{description}
%%%%%
\subsection{Bugs connus}
Y en a :(
