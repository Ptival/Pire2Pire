\chapter{Etat du projet}

%%%%%
\section{Caractéristiques}
\begin{itemize}
    \item Fonctionne sous GNU/Linux (Debian/Ubuntu/Fedora) et *BSD (FreeBSD)
    \item Démon Unix.
    \item Application multi-threadée
\end{itemize}
\section{Commandes implémentées}

\subsection{Commandes de protocole}
\begin{description}

    \item[list] Récupère la liste des fichiers présents sur les machines
    connues. Implémentée, sans options.

    \item[file fichier hash taille ip:port] Réponse d'un démon à la requête
    list. Implémentée seulement en TCP.

    \item[get hash début fin] Demande un bloc d'un fichier à un démon.
    Implémenté sans utiliser debut et fin.

    \item[ready hash délai ip port protocole début fin] Réponse à une requête
    get, en vue d'initier la connexion. Implémentée, mais ne se sert pas de
    début et fin.

    \item[neighbourhood] Demande une liste de voisins connus. Implémentée.

    \item[neighbour ip:port] Fournit l'adresse IP et le port TCP d'un des
    membres du réseau. Implémentée.

    \item[error] Message d'erreur. Implémentée.

\end{description}

\subsection{Commandes utilisateur}
\begin{description}

    \item[set] Fournit la liste des options disponibles. Aucune option n'est
    disponible...

    \item[help] Fournit la liste des commandes existantes. Implémentée.

    \item[list] Liste les fichiers présents sur les machines auxquelles le
    client est connecté. Aucune option n'est implémentée pour cette commande,
    elle retournera toujours la totalité des fichiers.

    \item[get hash] Télécharge le fichier dont le hash est passé en paramètre.
    Implémentée.

    \item[info] Donne des informations concernant le programme (nombre de
    machines connues, nombre de clients connectés).

    \item[download] Donne des informations sur les téléchargements en cours.

    \item[upload] Donne des informations sur les envois en cours. Implémentée
    en partie, ne calcule pas encore l'avancement.

    \item[connect ip:port] Se connecte à une autre machine. Implémentée.

    \item[raw ip:port cmd] Envoie la commande de protocole "cmd" à la machine
    dont l'ip/port sont passés en paramètre. Implémentée.

\end{description}

\section{Bugs connus}
\begin{itemize}
    \item Sensibles aux attaques par buffer overflow (utilisation massive de
strcpy, sprintf, etc.)
\end{itemize}
